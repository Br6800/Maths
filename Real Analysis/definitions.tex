\documentclass{article}
\usepackage[utf8]{inputenc}
\usepackage{amsmath}
\usepackage{amsthm}
\usepackage{amssymb}
\usepackage{graphics}
\usepackage{graphicx}
\usepackage{mathtools}
\usepackage{booktabs}
\DeclareMathOperator\dom{dom}
\DeclareMathOperator\eps{\epsilon}
\DeclareMathOperator\del{\delta}
\DeclareMathOperator\Del{\Delta}
\DeclareMathOperator\R{\mathbb{R}}
\DeclareMathOperator\N{\mathbb{N}}
\DeclareMathOperator\ran{ran}

\DeclareMathOperator\inti{\int_{a}^{b}}
\def\upint{\mathchoice%
    {\mkern13mu\overline{\vphantom{\intop}\mkern7mu}\mkern-20mu}%
    {\mkern7mu\overline{\vphantom{\intop}\mkern7mu}\mkern-14mu}%
    {\mkern7mu\overline{\vphantom{\intop}\mkern7mu}\mkern-14mu}%
    {\mkern7mu\overline{\vphantom{\intop}\mkern7mu}\mkern-14mu}%
  \int}
\def\lowint{\mkern3mu\underline{\vphantom{\intop}\mkern7mu}\mkern-10mu\int}
\title{Definitions}


\begin{document}
\maketitle
\newpage{}

\section*{Chapter 1}
\subsection{Definition 2*}
Peano axioms:
\begin{enumerate}
\item the number 1 exists
\item if a exists, then s(a) exists
\item if s(a) = s(b) then a=b
\item $s(a) \neq 1$ for all natural a
\item if T is a subset of N then if (1 is in T and for all x in T,
s(x) is in T) then T = N.
\end{enumerate}
\subsection{Definition 4*}
Total order axioms: \\
\newline
Let S be a set. A relation $\leq$ on S is called a total order if it satisfies
\begin{enumerate}
\item (TO1) For all s, $t \in S$ either s $\leq$ t or t $\leq$ s (or both);
\item (TO2) If s $\leq$ t and t $\leq$ s then s = t; and
\item (TO3) If r $\leq$ s and s $\leq$ t then r $\leq$ t.
\end{enumerate}
More succinctly,
\begin{enumerate}
\item (TO1) Given two numbers, one must be greater than or equal to the other
\item (TO2) squeeze inequality holds
\item (TO3) common sense inequality holds
\end{enumerate}
\subsection{Definition 5}
Boundedness (of subsets of sets with total order). \\
\newline
Let S be a set with total order $\leq$ (think N, Z, or Q) and suppose that
T $\subseteq$ S. We say that T is:
\begin{enumerate}
\item Bounded above in S if there exists s $\in$ S such that t $\leq$ s for all t $\in$ T. In this
case such an s is said to be an upper bound for T.
\item Bounded below in S if there exists s $\in$ S such that s $\leq$ t for all t $\in$ T. In this
case such an s is said to be an lower bound for T.
\item Bounded in S if T is bounded above and bounded below.
\end{enumerate}


\subsection{Definition 7}
Supremum and infimum. \\
\newline
Suppose that S is a set with total order $\leq$ and T $\subseteq$ S.
\begin{enumerate}
\item A least upper bound for T is an upper bound in S which is less than every
other upper bound. More precisely, $s_0$ is a least upper bound for T if whenever s
is an upper bound for T we have $s$ $\geq$ $s_0$ .
If $s_0$ actually belongs to T we say that
it’s a maximum or greatest element of T .
\item A greatest lower bound for T is a lower bound in S which is greater than every
other lower bound. More precisely, $s_0$ is a least upper bound for T if whenever s
is a lower bound for T we have $s_0$ $\geq$ $s$. If $s_0$ actually belongs to T we say that
it’s a minimum or least element of T .
\end{enumerate}
\subsubsection{infimum of $\frac{n+1}{n}$ = 1}
One is a lower bound obviously. So by Lemma 12 in the notes inf exists.
Suppose there is some lower bound $B$ greater than 1. Then by the archimedean
property, $\exists N \in \mathbb{N}$ such that $N>\frac{1}{B-1} \implies B > \frac{N+1}{N}$
which is a contradiction.
\subsection{Definition 13}
-Removed-

\subsection{Definition 22*}
The distance between two real numbers $x$ and $y$ is the number
d$(x, y)$ = $|x - y|$.

\subsection{Hidden Definition 1*}
There exists a set $\mathbb{R}$, called the real numbers with a total
order $\leq$ such that:
\begin{enumerate}
\item $\mathbb{Q}$ $\subset$ $\mathbb{R}$;
\item the total order $\leq$ on $\mathbb{R}$
restricts to the usual total order on $\mathbb{Q}$;
\item $\mathbb{R}$ has the supremum property:
every nonempty subset of $\mathbb{R}$ which is bounded
above has a least upper bound in $\mathbb{R}$; and
\item for all $x$ $\in$ $\mathbb{R}$ the set $\{ q \in \mathbb{Q} : q \leq x \}$ is bounded above, with least upper bound
equal to $x$.
\end{enumerate}
This set $\mathbb{R}$ carries arithmetic operations which extend those on $\mathbb{Q}$.
\newpage{}
\section*{Chapter 2}
\subsection{Definition 24}
A sequence of real numbers is a function $a : \mathbb{N} \rightarrow \mathbb{R}$. It is a standard
notational convention, when dealing with sequences, to write $a_n$ for the value $a(n)$ $\in$ $\mathbb{R}$
taken by the function a at $n$ $\in$ $\mathbb{N}$.


\subsection{Definition 25*}
Let a = $(a_n)^{\infty}_{n=1}$ be a sequence of real numbers. We say that the sequence
a converges if: \\
\newline
There is an L $\in \mathbb{R}$ with the property that for every $\epsilon$ $>$ 0 there exists
N $\in$ $\mathbb{N}$ such that for all n $\geq$ N , we have $|a_n - L|$ $<$ $\epsilon$.
If such an L exists we call it a limit of the sequence $(a_n)^{\infty}_{n=1}$ and say that the sequence
converges to L.
\subsubsection{We claim $a_n = \frac{3n^3-2n^2+16}{14n^3} \rightarrow \frac{3}{14}$.}
\begin{proof}
Fix $\epsilon>0$. By the archimedean property, there exists $N \in \mathbb{N}$ such that $N_1>\frac{1}{\epsilon} \implies \frac{1}{N_1}<\epsilon$. Choose $N=\max \{N_1, 3 \}$. We have:
\begin{align*}
\left |  a_n-\frac{3}{14} \right | &= \left | \frac{3n^3-2n^2+16}{14n^3}-\frac{3}{14} \right | \\
&= \left | \frac{-2n^2+16}{14n^3} \right | \\
&= \frac{|-2n^2+16|}{14n^3} \\
&= \frac{2n^2-16}{14n^3} \text{ Since $n \geq 3$ and $|x|=-x$ for $x \leq 0$} \\
&\leq \frac{2n^2}{2n^3} = \frac{1}{n} \\
&< \epsilon \quad \text{ Since $N \geq N_1$}
\end{align*}
\end{proof}
\subsubsection{We claim $a_n = \frac{3n^3-n}{14n^3} \not \rightarrow \frac{4}{14}$.}
\begin{proof}
Fix $N\in\mathbb{N}$. We claim $\exists n \geq N$ such that $|a_n-\frac{4}{14}|\geq\frac{1}{14}$. We have:
\begin{align*}
\left |  a_N-\frac{4}{14} \right | &= \left | \frac{3N^3-N}{14N^3}-\frac{4}{14} \right | \\
&= \left | \frac{-N^3-N}{14N^3} \right | \\
&= \frac{N+N^3}{14N^3} \\
&\geq \frac{1}{14}. \\
\end{align*}
\end{proof}
\subsection{Definition 27*}
A sequence $a = (a_n)^{\infty}_{n=1}$ is said to diverge if it does not converge to any
L. That is, for all L $\in$ $\mathbb{R}$ there exists an $\epsilon_0$ $>$ 0 such that for all N $\in$ $\mathbb{N}$ there is n $\geq$ N with
$|a_n - L|$ $\geq$ $\epsilon_0$ .
\subsubsection{The sequence $a_n = (-1)^n n$ diverges}
\begin{proof}
Fix $L \in \mathbb{R}, N_1 \in \mathbb{N}$. By the archimedean property, there exists a natural number
$N > \max \{ N_1, |L|+1 \}$.
 We have $N \geq N_1$ and
\begin{align*}
\left |  a_N-L \right | &= \left |N(-1)^N-L \right | \\
&\geq \left |N \right | - \left | L \right |\\
&\geq 1.
\end{align*}
\end{proof}
\subsubsection{The sequence $a_n = (-1)^n$ diverges}
\begin{proof}
Fix $L \in \mathbb{R}$ and suppose for contradiction that $a_n \to L$. Then
 $\exists N \in \mathbb{N}$ such that $\forall n \geq N$ we have $|a_n-L|<1$. So
\begin{align*}
\left |  a_{2N}-L \right |<1 &\implies \left |(-1)^{2N}-L \right |<1 \\
&\implies \left |1-L \right | <1\\
&\implies 1-L<1 \\
&\implies 0<L \\
&\text{and} \\
\left |  a_{2N+1}-L \right |<1 &\implies \left |(-1)^{2N+1}-L \right |<1 \\
&\implies \left |-1-L \right | <1\\
&\implies L+1<1 \\
&\implies L<0. \\
\end{align*}
So we have a contradiction. Since $L$ was fixed, $a_n \not \to L \quad \forall L \in \mathbb{R}$.
\end{proof}
\subsubsection{The sequence $a_n = (1-\frac{1}{n})(-1)^n$ diverges}
\begin{proof}
We claim that the sequence has two subsequences that do not converge to the same limit, so by the
contrapositive of proposition 47, the sequence a diverges. Let the first subsequence $(a_{n_k})_{k \in \mathbb{N}}$
be given by $k \mapsto n_k$, $n_k = 2k$. We claim $a_{n_k} \to 1$ as $n \to \infty$. Fix $\epsilon>0$. We have:
\begin{align*}
\left |  a_{n_k}-1 \right | &= \left |(1-\frac{1}{2k})(-1)^{2k}-1 \right | \\
&= \left |(1-\frac{1}{2k})-1 \right | \\
&= \frac{1}{2k}
\end{align*}
By the archimedean property, $\exists N \in \mathbb{N}$ such that $N>\frac{1}{2\epsilon}
\implies \epsilon > \frac{1}{2N}$. So for each $k\geq N$, $\epsilon>\frac{1}{2N}\geq\frac{1}{2k}=|a_{n_k}-1|$.
Since $\epsilon$ was fixed, we have $a_{n_k} \to 1$ as $n \to \infty$. Similarly,
for $(a_{m_k})_{k \in \mathbb{N}}$ given by $k \mapsto m_k$, $m_k = 2k+1$, we have
$a_{m_k} \to -1$ as $n \to \infty$, so we have demonstrated that
there are two subsequences of $a$ converging to a
different limit, so $a$ does not converge, i.e. $a$ diverges.
\end{proof}
\subsubsection{The sequence $a_n = \sin(n)$ diverges}
\begin{proof}
Yeah nah.
\end{proof}
\subsection{Definition 28*}
Let a be a sequence of real numbers. We write $a_n$ $ \to  $ $\infty$ and say that
the sequence a diverges to $\infty$ if for every K $\in$ $\mathbb{R}$ there exists N $\in$ $\mathbb{N}$ such that n $\geq$ N
implies $a_n$ $>$ K. We say that a diverges to -$\infty$ and write an $ \to  $ -$\infty$
if for every K $\in$ $\mathbb{R}$
there exists N $\in$ $\mathbb{N}$ such that n $\geq$ N $\Longrightarrow$ $a_n$ $<$ K.
\subsubsection{The sequence $a_n = n$ diverges to infinity}
\begin{proof}
Fix $M \in \mathbb{R}$. By the archimedean property, there exists a natural number $N>M$
and hence $\forall n \geq N$, $a_n = n \geq N >M$.
\end{proof}
\subsubsection{The sequence $a_n = \log(n)$ diverges to infinity}
\begin{proof}
Fix $M \in \mathbb{R}$. By the archimedean property, there exists
a natural number $N>e^M \implies \log(N)>M$
and hence $\forall n \geq N$, $a_n = \log(n) \geq \log(N) >M$.
\end{proof}
\subsubsection{Lemma 29 - divergent to infinity implies divergent}
\begin{proof}
Fix $L \in \mathbb{R}$, $N \in \mathbb{N}$. Since $a_n \to \infty$, $\exists N_1 \in \mathbb{N}$
such that $\forall n \geq N_1$, $a_n>\max \{ 1, 1+L \}>0$. Choose $n = \max \{N,N_1 \}$.
Then $n \geq N$ and $n \geq N_1 \implies a_n>1+L \implies |a_n-L| \geq 1$, so $a$ diverges.
\end{proof}
\subsection{Definition 33*}
A sequence $(a_n)^{\infty}_{n=1}$ is bounded if there exists B $\in$ $\mathbb{R}$ such that $|a_n |$ $\leq$ B
for all n $\in$ $\mathbb{N}$.

\subsection{Definition 37}
The annoying terminologies.
\begin{enumerate}
\item Monotonically increasing: $a_{n+1}>a_n$
\item nondecreasing: $a_{n+1} \geq a_n$
\item Monotonically decreasing: $a_{n+1}<a_n$
\item nonincreasing: $a_{n+1} \leq a_n$
\end{enumerate}
$\forall n \in $ $\mathbb{N}$. Note that sum of likewise monotones is likewise
monotone, sum of opposite monotones can be anything, product of monotones can be anything
product of increasing positive is increasing.

\subsection{Definition 41*}
A sequence a is cauchy if: \\
\newline
For every $\epsilon>0$, $\exists N \in $ $\mathbb{N}$ such that when
$m,n \geq N$ we have $|a_m-a_n|<\epsilon$.

\subsection{Definition 46*}
Let $k \mapsto n_k$ be a Monotonically increasing sequence of natural numbers.
Then the sequence $(a_{n_k})^{\infty}_{k=1}$ is called a subsequence of a.

\subsection{Definition 49*}
If $(a_n)$ is a sequence in $\mathbb{R}$, then the series generated by a is he sequence
$(s_k)$ defined by \begin{align*}
s_1 \vcentcolon = a_1 \\
s_2 \vcentcolon = s_1 + a_2 = a_1 +a_2 \\
\cdots \\
s_k \vcentcolon= s_{k-1} + a_k
\end{align*}
We call $a_n$ the terms of the series and $s_k$ the partial sums of the series.
\[
\sum_{n=1}^{\infty} a_n \vcentcolon= \lim_{k \to \infty}s_k.
\]
If the limit converges, we call it the sum of the series and if it diverges,
we say the series diverges.
\newpage{}
\section*{Chapter 3}
\subsection{Definition 53*}
A function on $\mathbb{R}$ is a function $f: \dom (f) \to $ $\mathbb{R}$, $\dom (f) \subseteq$ $\mathbb{R}$.
The range of $f$ is given by \[
\ran(f) = \{ y \in \mathbb{R} : y = f(x), \, x \in \dom(f) \}.
\]

\subsection{Definition 55}
The natural domain of a formula specifying a function is the largest subset of $\mathbb{R}$
on which the formula makes sense.

\subsection{Definition 58*}
A nonempty set of numbers $I\subseteq\mathbb{R}$ is an interval if whenever
$x,y \in I$ with $x<y$ then for each a such that $x<a<y$, $a \in I$.

\subsection{Definition 61*}
A function $f: \dom (f) \to $ $\mathbb{R}$ is defined near a point a $ \in \mathbb{R}$
if there is $\delta_0$ such that \[
\{ x:0<|x-a|<\delta_0 \} \subseteq \dom(f).
\]
In order for a function to be defined near a, it needs to be defined for all values
arbitrarily close to a on both sides.
\subsection{Definition 64*}
Suppose that $f$ is defined near a. $\{$(The values of $f$ converge to a limit
$L$) $\Longleftrightarrow$ $(f(x) \to L)$ as $x \to a$ $\}$ $\Longleftrightarrow \{ \lim_{x \to a}f(x) = L \}$ if for each $\epsilon>0$, $\exists \delta > 0$ such that $\forall x$, if \[
0<|x-a|<\delta \Longrightarrow |f(x)-L| < \epsilon.
\]

\subsection{Definition 68*}
A function $f$ is defined near but to the right of a if $\exists \delta_0>0$ such that
$(a, a+\delta_0) \subseteq \dom(f)$. If $f$ is defined near but to the right of a, then $f$
converges to $L$ as $x \to a^+$ (from the right) if for each $\epsilon > 0, \exists \delta > 0$ such that \[
0<x-a<\delta \Longrightarrow |f(x)-L| < \epsilon.
\]
Similarly, a function is defined near to the left of a, i.e.
$\exists \delta_0>0$ such that
$(a-\delta_0, a) \subseteq \dom(f)$. And if $f$ is defined to the left, then
$f$ converges to $L$ as $x \to a^-$ (from the left) if for each $\epsilon >0, \exists \delta >0$ such that \[
0<a-x<\delta \Longrightarrow |f(x)-L|<\epsilon.
\]

\subsection{Definition 73*}
Suppose $f(x)$ is defined on the interval $(a, \infty)$. Then $f(x) \to L$
as $x \to \infty$ if for each $\epsilon > 0, \exists N>0$ such that \[
x>N \Longrightarrow |f(x)-L|<\epsilon.
\]
Similarly, suppose $f(x)$ is defined on the interval $(-\infty, a)$. Then $f(x) \to L$
as $x \to -\infty$ if for each $\epsilon > 0, \exists N<0$ such that \[
x<N \Longrightarrow |f(x)-L|<\epsilon.
\]
\subsection{Definition 76*}
Let $f$ be defined near $a \in \dom(f)$. Then $f$ is continuous at a if \[
\lim_{x \to a}f(x) = f(a),
\]
i.e. if for each $\epsilon>0, \exists \delta > 0$ such that
$|x-a|<\delta \Longrightarrow |f(x)-f(a)|<\epsilon.$

\subsection{Definition 78}
A function $f$ is continuous if it is continuous at each $a \in \dom(f)$.

\subsection{Definition 83*}
$f(I) \vcentcolon = \{ f(x) : x \in I \}$. \\
\newline
Let $f: I \to \mathbb{R}$ be a function. Then $f$ is bounded on $I$ if $f(I)$ is bounded, i.e.
$\exists B$ such that $|f(I)|\leq B$.

\subsection{Definition 89*}
Let $f:[a,b] \to \mathbb{R}$. Then $f$ is continuous from above if $\lim_{x \to a^+}f(x) = f(a)$, i.e.
for each $\epsilon>0, \exists \delta>0$ such that \[
x \in [a, a+\delta) \Longrightarrow |f(x)-f(a)|<\epsilon.
\]
Similarly, $f$ is continuous from below if $\lim_{x \to a^-}f(x) = f(a)$, i.e.
for each $\epsilon>0, \exists \delta>0$ such that \[
x \in (a-\delta, a] \Longrightarrow |f(x)-f(a)|<\epsilon.
\]
\newpage{}
\section*{Chapter 4}
\subsection{Definition 96*}
A function $f:I \to \mathbb{R}$ is uniformly continuous on $I$ if for each
$\epsilon>0, \exists \Delta(\eps)>0$ such that \[
|x-y|<\Delta \Longrightarrow |f(x)-f(y)|<\epsilon \quad \forall x, y\in I.
\]
\subsubsection*{Not U.C. (importantly) on $I$. A single point won't work.}
Typically you just want the sequential characterisation to prove dicontinuity at a point. Fix $\Delta>0$. We claim
$\exists x(\Del),y(\Del) \in \dom(f)$ such that $|x-y|<\Delta$ and $|f(x)-f(y)| \geq 1$.
\subsubsection*{$f:[0,\infty), f(x) = x^2$ not U.C.}
Fix $\Del>0$. Choose $y \geq \frac{1}{\Del}$, $x = y+\Del \slash 2$. Then $|x-y| = |\Del \slash 2|<\Del$ and
\begin{align*}
f(x)-f(y) &= x^2-y^2 \\
&= (y+\Del \slash 2)^2-y^2 \\
&= \Del^2 \slash 4 + y\Del \\
&> y\Del \\
&\geq 1.
\end{align*}
so $|f(x)-f(y)|>1$ and we're done.
\newpage{}
\subsection{Definition 101*}
Suppose we have a sequence of functions $(f_n)_{n\geq 1}$ where
$f_n:I \to \mathbb{R}$ for all $n$ and a function $f:I \to \mathbb{R}$.
Then we say that $f_n$ converges to $f$ pointwise if $f_n(x) \to f(x)$ as $n \to \infty$
for each $x \in I$. That is for each $x$, $\forall \epsilon > 0, \exists N(x, \eps) \in \mathbb{N}$ such that
for $n>N$ we have $|f_n(x)-f(x)|<\epsilon$. \\
\newline
Note that $f(x) \Longleftrightarrow L$ and $f_n \Longleftrightarrow a_n$.
\subsubsection*{$x \slash (1+x^n)$ converges pointwise to $x$ for $1 > x \geq 0$, to $0$ for $x>1$.}
When $1>x \geq 0,$ $x \mapsto x^n \to 0$ as $n \to \infty$, so by the algebra of limits,
function converges to $x \slash 1 = x$.
\subsection{Definition 103*}
$f_n$ converges to $f$ uniformly on $I$ if $\forall \epsilon > 0, \exists \Pi(\eps) \in \mathbb{N}$ such that \[
n> \Pi \Longrightarrow |f_n(x)-f(x)|<\epsilon \quad \forall x \in I.
\]. More succinctly,
if $\forall \epsilon > 0, \exists \Pi \in \mathbb{N}$ such that \[
n> \Pi \Longrightarrow d_{\infty}(f_n,f)<\epsilon.
\].
\subsubsection*{$f_n \not \to f$ uniformly:}
Prove $f_n$ are continuous and $f$ isn't and use theorem 111.
\subsection{Hidden Definition 2*}
The distance between two functions $f$ and $g$ is the number $$d_\infty(f,g) = \sup_{x \in I}|f(x)-g(x)|$$
\subsection{Definition 107*}
For a set X, a metric on X is a function $d: X \times X \to [0, \infty)$
such that for each $x,y,z \in X$:
\begin{enumerate}
\item $d(x,y) = 0 \text{ iff } x=y$
\item $d(x,y) = d(y, x)$
\item $d(x,z) \leq d(x, y) + d(y, z)$
\end{enumerate}
We call the set X together with the metric $d$, $(X, d)$ a metric space.
\subsection{Definition 113*}
Let $(f_n)_{n=1}^{\infty}$ be a sequence of functions on an interval $I$.
Then the sequence $(f_n)$ is uniformly Cauchy in the interval $I$ if
for each $\epsilon > 0, \exists \Pi(\eps) \in \mathbb{N}$ such that \[
m,n > \Pi \Longrightarrow d_{\infty}(f_n, f_m)<\epsilon.
\]
\subsection{Definition 117*}
Let $C_b(I) \vcentcolon=\{ f:I \to \mathbb{R}:f \text{ is continuous and bounded} \} \subseteq B(I)$.
If $I$ is closed, then $C_b(I) = \{ f:I \to \mathbb{R}:f \text{ is continuous} \}$ by theorem 84.


\section*{Chapter 5}
\subsection{Definition 120}
Fix an interval $I$, a function $f:I \to \mathbb{R}$ and a point $x \in I$. Then
$f$ is differentiable at $x$ if \[
\lim_{h \to 0}\frac{f(x+h)-f(x)}{h} = f'(x) \in \mathbb{R}.
\]
\subsection*{Applications}
$1/x$ is differentiable for $x \neq 0$
\subsection{Definition 134}
Fix a function $f: \mathbb{R} \to \mathbb{R}$ that is continuous on an interval $[a,b]$.
We define \[
\int_{a}^{b}f(x)dx \vcentcolon= \lim_{n \to \infty}S_n(f) = \lim_{n \to \infty}\sum_{i=1}^{n} f \left( a+\frac{i}{n}(b-a) \right) \times \frac{b-a}{n}.
\]
\subsection{Definition 142}
A partition of a closed interval $[a,b]$ is a sequence of numbers in $\mathbb{R}$, $P \vcentcolon= (x_0,...,x_n)$
such that $a = x_0 <...<x_n = b$. We say $P$ is a finer partition of $[a,b]$ than $Q \vcentcolon= (y_0,...,y_m)$ if
there is a monotonically increasing function $j \mapsto i(j)$ from $\{0,...,m \}$ to $\{0,...,n \}$
such that $y_j = x_{i(j)} \quad \forall j \leq m$.
\subsection{Definition 143}
Fix a bounded function $f:[a,b] \to \mathbb{R}$ and a partition $P = (x_0,\hdots,x_n)$ of [a,b]. Then
the upper and lower Riemann sums of $f$ with respect to $P$ are given by: \begin{align*}
U_{P}(f) &= \sum_{i=1}^{n} \sup \{ f(x):x_{i-1} \leq x \leq x_i\} \times (x_{i} - x_{i-1}) \, \text{ and } \\
L_{P}(f) &= \sum_{i=1}^{n} \inf \{ f(x):x_{i-1} \leq x \leq x_i\} \times (x_i-x_{i-1}), \\
\end{align*}
respectively.
\subsection{Definition 146}
Fix a bounded function $f:[a,b] \to \mathbb{R}$. The upper and lower Riemann integrals of $f$
over $[a,b]$ are given by:
\begin{align*}
\upint_a^b f(x)\,dx &\vcentcolon= \inf \{ U_{P}(f):P \text{ is a partition of } [a,b] \} \\
   \lowint_a^b f(x)\,dx &\vcentcolon= \sup \{ L_{P}(f):P \text{ is a partition of } [a,b] \} \\
\end{align*}
respectively. If the two integrals agree, $f$ is Riemann integrable over $[a,b]$ and
$$ \int_a^bf(x)dx \vcentcolon= \upint_a^b f(x)dx.$$
\section*{Chapter 6 has no definitions.}
\section*{Chapter 7}
\subsection{}
Let $X \vcentcolon= (x_n)$ be a sequence in $\mathbb{R}$. Then the series
$\sum_{n=1}^{\infty}x_n$ is said to be absolutely convergent if $\sum_{n=1}^{\infty}|x_n|$ converges in
$\mathbb{R}$.
\subsection{}
Let $X \vcentcolon= (x_n)$ be a sequence of nonzero numbers in $\mathbb{R}$. Then $X$ is alternating if
the terms $(-1)^{n+1}x_n$ all have the same sign. If $X$ is alternating, then the series it generates,
$\sum_{n=1}^{\infty}x_n$ is said to be an alternating series.
\section*{Chapter 8}
\subsection{Definition 170}
Fix $D \subseteq \mathbb{R}$, $x \in D$. If $(f_n)$ is defined on $D$, the sequence of partial sums
$(s_k)$ of the infinite series $\sum_{n=1}^{\infty}f_n$ is defined by $s_k(x) = \sum_{n=1}^{k}f_n(x)$.
If $(s_k)$
converges on $D$,
then we say $\sum_{n=1}^{\infty}f_n
= f \text{ or } \lim_{k \to \infty}s_k = f$.
\subsection{Definition 176}
A series of functions $\sum_{n=0}^{\infty}f_n$ is a power series around
$x=c$ if $$f_n(x) = a_n (x-c)^n$$ with $a_n, c \in \R \quad \forall n \in \N_0$.
So a power series has the form: $$
\sum_{n=0}^{\infty} a_n(x-c)^n = a_0 + a_1(x-c)+a_2(x-c)^2+\hdots+a_n(x-c)^n+ \hdots
$$
\subsection{Hidden Definition 3}
Let $(b_n)$ be a bounded and nonnegative sequence. Then $$
\limsup_{n \to \infty}b_n \vcentcolon= \inf \{ \sup \{b_n: n \geq m\}:m \geq 1 \}
= \lim_{n \to \infty} \sup_{k \geq n} \{b_k \}.
$$
There are two relevant properties of the limsup:
\subsubsection*{If $L > \limsup_{n \to \infty}b_n$}
Then $\exists N \in \N$ such that for each $n \geq N$, $b_n < L$.
\subsubsection*{If $M < \limsup_{n \to \infty}b_n$}
Then $\exists$ infinitely many $n \in \N$ such that $b_n > M$.
\subsubsection*{It always exists:}
If $(b_n)$ is bounded, then $(\sup \{b_k:k \geq n \})_{n=1}^{\infty}$ is bounded and
nonincreasing by definition of the supremum,
so the limsup exists by the monotone convergence theorem. If not, it is supposedly just equal to infinity.
\subsection{Definition 177}
Fix a power series centred at $x=c$. Then
\subsubsection*{If the sequence $(|a_n|^{1 \slash n})$
is bounded:}
Set $\rho = \limsup_{n \to \infty}(|a_n|^{1 \slash n})$. It is often easier to
set $$\rho = \lim_{n \to \infty} \left| \frac{a_{n+1}}{a_{n}} \right|. $$
Typically it will be easier when factorials are involved.
It may come in handy that $\lim_{n \to \infty} n ^ {1 \slash n} = 1$.
\subsubsection*{If not:}
Set $\rho = + \infty$
\subsubsection*{The radius of convergence of $\sum_{n=1}^{\infty}a_n x^n $ is defined by:}
$$
R \vcentcolon=
\begin{cases}
0 & \rho = + \infty \\
1 \slash \rho & \rho \in (0, \infty) \\
+ \infty & \rho = 0
\end{cases}
$$
The interval of convergence is the open interval $(-R,R)$. A power series may converge
on the boundaries of the interval, this is a case by case phenomenon.
\subsection{Hidden Definition 4}
If and only if the sequence of remainders $(R_n(x)) \to 0$ as $n \to \infty$ in some interval $\{x:|x-x_0|<R\}$ , we say that
the power series $$
f(x) = \sum_{n=0}^{\infty} \frac{f^{(n)}(x_0)}{n!}(x-x_0)^n, \quad |x-x_0|<R
$$
is the Taylor series expansion of $f$ at $x_0$.
\end{document}
