\documentclass{article}
\usepackage[utf8]{inputenc}
\usepackage{amsmath}
\usepackage{amsthm}
\usepackage{amssymb}
\usepackage{graphics}
\usepackage{graphicx}
\usepackage{xcolor}
\usepackage{mathtools}
\usepackage{booktabs}
\DeclareMathOperator\dom{dom}
\DeclareMathOperator\eps{\epsilon}
\DeclareMathOperator\del{\delta}
\DeclareMathOperator\Del{\Delta}
\DeclareMathOperator\R{\mathbb{R}}
\DeclareMathOperator\N{\mathbb{N}}
\DeclareMathOperator\ran{ran}

\DeclareMathOperator\Fs{F_{\sigma}}
\DeclareMathOperator\Gd{G_{\delta}}

\DeclareMathOperator\inti{\int_{a}^{b}}

\AtBeginEnvironment{align}{\setcounter{equation}{0}}
\AtBeginEnvironment{gather}{\setcounter{equation}{0}}
\title{Counterexamples in Topology}
\begin{document}
\maketitle{}
\newpage{}
\section{Chapters 1 \& 2}
\subsection{Theory we found gold again}
\subsubsection{Axioms}
Let $T = (X, \tau)$ where $X$ is a set and $\tau$
is a collection of subsets of $X$; we call these subsets open sets. The collection
$\tau$ is said to be a topology for the set $X$ given that members of $\tau$ satisfy the axioms:
\begin{align*}
O_1: & \text{ The finite or infinite union of open sets is open} \\
O_2: & \text{ Let $A_1,\hdots,A_n$ be a
 collection of open sets in $\tau$ where $n \in \N$.
  Then their intersection $\bigcap_{i=1}^n A_i$ is also open} \\
O_3: & \text{ The sets $X$ and $\emptyset$ are open.}
\end{align*}
We define a set to be closed if its complement is an open set in $\tau$.
So by De Morgan's laws, we can swap intersections with unions to obtain
similar conditions for closed sets:
\begin{align*}
C_1: & \text{ The finite or infinite intersection of closed sets is closed} \\
C_2: & \text{ Let $A_1,\hdots,A_n$ be a
 collection of closed sets where $n \in \N$.
  Then their union $\bigcup_{i=1}^n A_i$ is also closed} \\
C_3: & \text{ The sets $X$ and $\emptyset$ are closed.}
\end{align*}
\subsubsection{Coarser, Weaker or Smaller}
We say $\tau_1$ is coaser than $\tau_2$ if all members of $\tau_1$ are in $\tau_2$
We say that $\tau_1 \leq \tau_2$ or $\tau_1 \supseteq \tau_2$. The coarsest topology in the indiscrete topology
and the finest topology is the discrete topology. The ordering $\leq$ only forms a partial ordering, not all topologies are comparable.
\subsubsection{Other Types of Sets}
An $\Fs$ set is a set equal to the countable union of closed sets. The complement of an $\Fs$ set is a
$\Gd$ set. Closed sets need not be $\Gd$ but are trivially $\Fs$.
% note that the intersection of a singleton is as expected. See proof wiki.
Clopen sets are trivially both $\Gd$ and $\Fs$.
\subsubsection{Neighbourhoods}
% related to open sets
A neighbourhood $N_A$ for a set $A$ is a subset of $X$ such that $$
A \subset O_A \subset N_A.
$$
We call it an open neighbourhood if $N_A$ is also open.
\subsubsection{Bases}
\paragraph{Subbase \& Base}
Any collection of subsets $\mathcal{S}$ can generate a topology for $X$ by taking all
sets formed from the union of finite intersections $S, \emptyset$ and $X$. If the union of the collection $S$ is equal to
the set $X$ and $$\forall s_1,s_2 \in \mathcal{S}, \,\, \forall a \in (s_1 \cap s_2)  \,\, \exists \,\,s_3 \in \mathcal{S}:a \in s_3 \subset (s_1 \cap s_2)$$
then $\mathcal{S}$ is a base for $\tau$. In this case
the topology can be generated more simply by taking all unions of members of
$\mathcal{S}$ as open sets, since each finite intersection
is equal to the union of some collection of open sets in $\mathcal{S}$
which contain each of its points. If two (sub)bases generate the same topology
they are equivalent.
\paragraph{Local Basis}
The neighbourhood basis $\tau$ for a point $x \in X$ is a collection of open neighbourhoods
$$ \{ N(x):\forall \,\,O: x \in O \,\, \exists \,\, A \in N(x) : A \subset O \}$$
%
% this is a bit weird, N(x) is already a collection of sets.
% by defn, x is in A
%
Suppose $(X,\tau)$ is a topological space.
For any $Y \subset X$ we can generate a topology $\tau_Y$ by
 defining open sets to be equal
to the intersection of $Y$ and a member of $\tau$. The pair $(Y,\tau_Y)$ is called a
subspace of $(X,\tau)$ and $\tau_Y$ is called the induced topology for $Y$. A set $U \subsetneq Y$
has a property relative to $Y$, for example open relative to $Y$, if $U$ has the property in the subspace $(Y,\tau_Y)$.
If this holds for any subspace, we say the property is hereditary. If this holds
for closed subsets and their induced topology,
we say the property is weakly hereditary. Compactness is one such weakly hereditary property.
\subsubsection{Compactness}
A space $X$ is compact if for each open cover, a collection of open subsets whose union contains $X$, there
is a finite subcollection whose union also contains $X$. Suppose $X$ is compact and $\{O_\alpha\}$ is an open cover for a closed subset $Y$.
Then the union of $X-Y$ (which is open by definition) and $\{O_\alpha\}$ covers $X$. Since $X$ is compact, there exists a finite \textcolor{red}{subcollection} of $\{X-Y, O_\alpha\}$ that also covers $X$.
Either $X-Y$ is in the subcollection or it isn't.
In any case, omitting it from the subcollection will give a new
finite subcollection of $\{O_\alpha\}$ that covers $Y$ ($X-Y$ need not be in $\{O_\alpha\}$).
So a closed subset of  a compact space is compact.
A compact subset of a compact space need not be closed.

\subsubsection{Limit Points}
A limit point $p$ is a limit point of a set $A$
if every open set containing $p$ also contains
a different point from $A$. If that point is not required to be distinct from $p$,
the point is called an adherent point.
\paragraph{$\omega$-accumulation points}
Are a special case of limit points where
 every open set that contains $p$ must contain infinitely many points from $A$.
\paragraph{Condensation points}
Contain uncountably many points from $A$.
\paragraph{Limit point of a sequence}
A point $p_s$ is a limit point \textbf{for the sequence}
 if every open set containing $p_s$ contains all but finitely many points of the sequence. Every limit
 point of a sequence is also an accumulation point of that sequence, every open set containing $p_s$ contains infinitely many terms in the sequence.
\paragraph{Set associated with a sequence}
Limit points of a sequence need only be adherent points of the associated set of their elements, take the discrete topology
with two points and a sequence of one point and the second point repeating.
\paragraph{Distinct Sequences}
Every accumulation point of a distinct sequence is an $\omega-$accumulation point of the associated set. Also, every sequence corresponding to a set with an $\omega-$accumulation point has that point
as an accumulation point.

\paragraph{Isolated Points \& Derived Sets}
An isolated point is any point not in the derived set of $A$. Such points are
isolated because they are contained in an open set with no other points of $A$. If a set
has no isolated points it is said to be dense in itself. If it is also closed, it is perfect. By definition, closed
sets contain all their limit points since the complement is open and has no points of the set.
Suppose a set $A$ contains each of its limit points $x$.
We claim the set is closed because the complement set $X-A$ is open.
Since $x \in X-A$ is not a limit point of $A$, it must be that
$$
\exists O_x: \{x\} \subset O_x \subset X-A.
$$
Where $O_x$ is chosen to be an open set that contains $x$ but no points in $A$. By the
open-set-neighbourhood equivalence, $X-A$ is open. So \textbf{a set is perfect iff it equals its derived set}.
\subsubsection{Closures, Interiors, regular sets and connected sets}
The closure of a set is the set in union with its limit points. Equivalently,
the closure of a set $A$ is the smallest closed set $A^-:A \subset A^-$. Similarly,
the interior of $A$ is the largest open set $A^\circ:A^\circ \subset A$. By definition, $A^{'-}$ is the smallest
closed set containing $X-A$. We claim $ A^{'-'}$ is the largest open set in $A$, i.e. $A^\circ = A^{'-'}$.
Suppose not. Then there must be some larger open set which is a subset of $A$. But then it's complement in $X$ must be a closed set
containing $X-A$ that is smaller than $A^{'-}$. Contradiction. \\ \newline From the axioms, $A^\circ$ is the union of all open sets in $A$. The closure and complement
operations admit at most fourteen different variant sets when applied to $A$.
\newpage{}
Notice the reversal in these relations:
$$
\bigcup_{i=1}^n A_i^- = \big( \bigcup_{i=1}^n A_i \big)^-, \quad \bigcup_{i=1}^\infty A_i^- \subset \big( \bigcup_{i=1}^\infty A_i \big)^- \quad \forall n \in \N.
$$

$$
\big( \bigcap_{i=1}^n A_i \big)^\circ = \bigcap_{i=1}^n A_i^\circ, \quad \big( \bigcap_{i=1}^\infty A_i \big)^\circ \subset \bigcap_{i=1}^\infty A_i^\circ \quad \forall n \in \N.
$$


$$
\big( \bigcup_{i=1}^n A_i \big)^e = \bigcap_{i=1}^n A_i^e, \quad \big( \bigcup_{i=1}^\infty A_i \big)^e \subset \bigcap_{i=1}^\infty A_i^e \quad \forall n \in \N.
$$


$$
\bigcup_{i=1}^n A_i^e = \big( \bigcap_{i=1}^n A_i \big)^e, \quad \bigcup_{i=1}^\infty A_i^e \subset \big( \bigcap_{i=1}^\infty A_i \big)^e \quad \forall n \in \N.
$$


An open set is regular open if $A=A^{-\circ}$.
Its complement is regular closed because it satisfies
$(A^{'}) = (A^{'})^{\circ -}$. \textbf{Regular sets} don't obey all the rules for axiomatic sets.
We only have the finite parts of the axioms: $$
\bigcap_{i=1}^{n} {\bf O_i} \equiv {\bf O}, \quad \bigcup_{i=1}^n { \bf C_i} \equiv {\bf C}.
$$
We define the boundary of a set $A$:
$$
A^b \vcentcolon= \{a \in A: a \not \in A^\circ, \,\, a \in A^- \} = A^- - A^\circ = A^- \cap (X-A)^-.
$$
A set is closed iff the boundary is a subset, a set is open iff it is disjoint from its boundary and
a set is clopen iff the boundary is empty. The boundary itself is closed as the intersection of two closed closures. In general, $A^{bb} \subset A^b$. For example,
take trivial topology on $\{1,2,3,4\}$, $A=\{1,2\}$. We have $$
A^- = X = (X-A)^- = A^- \cap (X-A)^- = A^b = A^{b-}, \quad A^{b-} \cap (X-A^{b})^- = A^{bb} = X \cap \emptyset^- = X \cap \emptyset = \emptyset.
$$
We define the exterior of a set $A$:
$$
A^e \vcentcolon= A^{'\circ} = A^{-'}.
$$
We have $$
A^\circ \subset A^{ee}.
$$
Two sets $A,B$ are separated if $$
A^- \cap B = B^- \cap A = \emptyset.
$$
A set $A$ is connected if it is not the union of any two separated sets in $X$.
\subsubsection{}


\subsubsection{}


\subsubsection{}

\subsubsection{}

\subsubsection{}
\subsection{Examples}
\subsubsection{Discrete \& Indiscrete Topologies}
\paragraph{Discrete Topology}
The discrete topology on $X$ is defined by taking all subsets of $X$ to be
open and hence, since their complements in $X$ are in $X$, closed.
The discrete topology will be categorised by one of the following:
\begin{gather}
\text{If $X$ is finite,
we call it a finite discrete topology} \\
\text{If $X$ is countably infinite,
we call it a countable discrete topology} \\
\text{If $X$ is uncountably infinite,
we call it an uncountable discrete topology.}
\end{gather}
\paragraph{Indiscrete Topology}
The trivial topology takes the only open sets to be $\emptyset$ and $X$. All subsets other than
$X$ or $\emptyset$ are trivially neither open nor closed.
\subsubsection{Particular \& Excluded Point Topologies}
We generate this topology by defining open sets to be equal to $\emptyset$ or subsets of $X$ that contain some $p \in X$.
Suppose we have another topology generated by defining open sets to be those sets different from $X$ that do not contain $p$. Then clearly the two topologies are not comparable
since the collections do not share any members other than $X$ and $\emptyset$ and both have additional members since $X$ has at least two elements by definition.
\subsubsection{Cofinite Topology}
The cofinite topology on any set $X$ is the collection $$
\tau = \{ A \subset X : A = \emptyset \text{ or } X \backslash A \text{ is finite} \}.
$$
It follows that all closed subsets are finite sets or the entire set $X$. Suppose $X$ is
uncountable. Then since the complement of any $A \in \tau$ in $X$ is finite, we must have that $A$
is uncountable. This implies that there exist sets that are open but not $\Fs$. The complement of an open set is closed
by definition and so there exist closed sets that are not $\Gd$, since the complement of a $\Gd$ set must be $\Fs$.
\subsubsection{A set is open iff it is a neighbourhood of each point}
Suppose that a set $A \subset X$ is a neighbourhood for each of its points. Then
$\forall a \in A$,
$$
\{ a \} \subset O_a \subset A.
$$
So $A$ is the union of a collection of open sets and by $O_1$ the set $A$ is open. The converse follows from the definition.
\paragraph{Example}
In particular for $\R^n$ under the usual topology we have that
a set $\Omega \subset \R^n$ is open iff for each $x \in \Omega,$
$$
\exists S(x,r),r>0:S(x,r) = \{y \in \R^n: |x-y|<r \} \subset \Omega.
$$

\subsubsection{Equivalent Bases}
% X = {1,2,3} B1 = {1,2}, {2,3}, {1,3} and B2 = {1}, {2}, {3}.
\subsubsection{Compact Subsets need not be closed}
In the trivial topology, all subsets are compact but only $X$ and $\emptyset$ are closed. Suppose $(Y,\tau_Y)$ is
an infinite subspace of the cofinite topology and $\{U_\alpha\}$ is an open cover for $Y$. Then any $A \in \{U_\alpha\}$ is open and hence
has a finite complement in $Y$. So the collection $\{A_1,\hdots,A_n,A \}$ where $A_1,\hdots, A_n$ are open sets from $\{U_\alpha\}$ that contain each point is a finite subcollection that covers $Y$.
So every subspace of a cofinite topological space is compact (proof if $Y$ is finite is trivial). Since closed sets are finite in the cofinite topology, $Y$ isn't closed.

\subsubsection{Limit Points}
\paragraph{A sequence can have uncountably many limit points $p_s$:}
Any sequence in a trivial topology.
\paragraph{Accumulation points of a sequence need not be limit points of the sequence \& limit points of a set need not be limit points of a sequence}
Consider $[-1,1]$ with open sets $(a,b):a<0,b>0$. This is called the overlapping interval topology.
Zero is an accumulation point of the sequence $0,0.5,0,0.5\hdots$ but not a limit point of that sequence.
The point $0.25$ is a limit point of $[-1,1]$ but not a limit point of the sequence.

\subsubsection{Closures \& Interiors}
\paragraph{A set in union with $\omega-$accumulation points or condensation points need not be closed}
Try the trivial topology with a finite subset.r

\paragraph{Interior is contained in the double exterior}
The right half-open interval topology on $\R$ is defined by taking open sets of the form $[a,b)$.
The following results are trivial:
\begin{gather*}
(-\infty, a) \equiv C \iff [a, \infty) \equiv O \\
[a, \infty) \equiv C \iff (-\infty,a) \equiv O \\
\implies (-\infty, a) \equiv CO, \quad [b, \infty) \equiv CO \\
\implies [a,b) \equiv CO.
\end{gather*}
From the axioms we find that
\begin{gather*}
(a,b) = \bigcup_{n=1}^\infty [a+1/n,b+1/n) \equiv O \\
(a,\infty) = \bigcup_{n=1}^\infty (a,a+n) \equiv O.
\end{gather*}
Take the set $(3,4)$.
We have $(3,4)^{\circ} = (3,4)$, ${(3,4)}^{-} = [3,4) \implies {(3,4)}^{e} = {[3,4)}^{'}$.
But $[3,4)^{'} = [3,4)^{'-} \implies (3,4)^{ee} = [3,4)^{''} = [3,4)$. Hence $(3,4)^\circ \subsetneq (3,4)^{ee}$.



\end{document}
