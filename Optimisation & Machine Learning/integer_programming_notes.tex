%\begin{center}
%\includegraphics[scale=0.05, angle=-90]{figboga/remark.JPG}
%\end{center}
\documentclass{article}
\usepackage[utf8]{inputenc}
\usepackage{amsmath}
\usepackage{hyperref}
\usepackage{amsfonts}
\usepackage{listings}
\usepackage{color}
\definecolor{dkgreen}{rgb}{0,0.6,0}
\definecolor{gray}{rgb}{0.5,0.5,0.5}
\definecolor{mauve}{rgb}{0.58,0,0.82}
\lstset{frame=tb,
  language=Python,
  aboveskip=3mm,
  escapechar=\%,
  belowskip=3mm,
  showstringspaces=false,
  columns=flexible,
  basicstyle={\small\ttfamily},
  numbers=none,
  numberstyle=\tiny\color{black},
  keywordstyle=\color{black},
  commentstyle=\color{dkgreen},
  stringstyle=\color{black},
  breaklines=true,
  breakatwhitespace=true,
  tabsize=3
}
\usepackage{ mathrsfs }
\usepackage{amsthm}
\usepackage{amssymb}
\usepackage{graphics}
\usepackage{graphicx}
\usepackage{mathtools}
\usepackage{booktabs}
\DeclareMathOperator\dom{dom}
\DeclareMathOperator\vphi{\varphi}
\DeclareMathOperator\eps{\epsilon}
\DeclareMathOperator\del{\delta}
\DeclareMathOperator\Del{\Delta}
\DeclareMathOperator\lm{\lambda}
\DeclareMathOperator\deq{\vcentcolon=}
\DeclareMathOperator\R{\mathbb{R}}
\DeclareMathOperator\pr{\mathbb{P}}
\DeclareMathOperator\Z{\mathbb{Z}}
\DeclareMathOperator\N{\mathbb{N}}
\DeclareMathOperator\F{\mathbb{F}}
\DeclareMathOperator\A{\mathbb{A}}
\DeclareMathOperator\HH{\mathbb{H}}
\DeclareMathOperator\minn{\text{Minimise} \quad }
\DeclareMathOperator\maxx{\text{Maximise} \quad }
\DeclareMathOperator\st{\text{Subject to} \quad }
\DeclareMathOperator\nc{\text{no constraints}}
\DeclarePairedDelimiter\ceil{\lceil}{\rceil}
\DeclarePairedDelimiter\floor{\lfloor}{\rfloor}
\DeclareMathOperator\bx{\bold{x}}
\DeclareMathOperator\by{\bold{y}}
\DeclareMathOperator\bu{\bold{u}}
\DeclareMathOperator\bs{\bold{s}}
\DeclareMathOperator\id{\text{Id}}
\DeclareMathOperator\bb{\bold{b}}
\DeclareMathOperator\bA{\bold{A}}
\DeclareMathOperator\bp{\bold{p}}
\DeclareMathOperator\bc{\bold{c}}
\DeclareMathOperator\C{\mathbb{C}}
\DeclareMathOperator\ran{ran}
\DeclareMathOperator\img{Im}
\DeclareMathOperator\op{\oplus}
\DeclareMathOperator\ot{\otimes}
\DeclareMathOperator\diam{diam}
\DeclareMathOperator\ite{int}
\DeclareMathOperator*{\argmax}{arg\,max}
\DeclareMathOperator*{\argmin}{arg\,min}
\DeclareMathOperator\cd{card}
\DeclareMathOperator\la{\langle}
\DeclareMathOperator\ra{\rangle}
\DeclareMathOperator\erf{erf}
\DeclareMathOperator\erfc{erfc}
\DeclareMathOperator\iso{\,\simeq\,}
\DeclareMathOperator{\sgn}{sgn}
\DeclareMathOperator{\lcm}{lcm}

\newcommand{\an}[1]{\langle \, #1 \, \rangle}
\newcommand{\GL}{\text{GL}(\mathbb{R}^n)}
\newcommand{\GLM}{\text{GL}(n,\mathbb{R})}
\newcommand{\quo}{/ _\sim}
\newcommand{\phii}{\phi^{-1}}
\newcommand{\pa}{\partial}
\newcommand{\lb}{\left}
\newcommand{\rb}{\right}
\newcommand{\aps}{\alpha_S}
\newcommand{\apl}{\alpha_L}
\newcommand{\tdd}{\frac{d^2T}{dx^2}}
\newcommand{\dx}{\frac{d}{dx}}
\newcommand{\seq}{(x_n)_{n \geq 1}}
\newcommand{\sseq}{(x_{n_k})_{k \geq 1}}
\newcommand{\fseq}{(f_n)_{n \geq 1}}
\newcommand{\elll}{\ell^{\infty}(\N)}
\newcommand{\norm}{{\|.\|}}
\newcommand{\inner}{\langle .,. \rangle}
\newcommand*\dd{\, \mathop{}\!\mathrm{d}}
\newcommand*\DD[1]{\, \mathop{}\!\mathrm{d^#1}}
\newcommand{\Ga}[1]{\frac{1}{\sqrt{2\pi\sigma^2}} e^{-\frac{\left(#1\right)^2}{2\sigma^2}}}
\usepackage{color}
%\title{}
\newcommand*\autoop{\left(}
\newcommand*\autocp{\right)}
\newcommand*\autoob{\left[}
\newcommand*\autocb{\right]}
\DeclareRobustCommand*\{{\ifmmode \left\lbrace \else \textbraceleft \fi }
\DeclareRobustCommand*\}{\ifmmode \right\rbrace \else \textbraceright \fi }
\AtBeginDocument {%
   \mathcode`( 32768
   \mathcode`) 32768
   \mathcode`[ 32768
   \mathcode`] 32768
   \begingroup
       \lccode`\~`(
       \lowercase{%
   \endgroup
       \let~\autoop
   }\begingroup
       \lccode`\~`)
       \lowercase{%
   \endgroup
       \let~\autocp
   }\begingroup
       \lccode`\~`[
       \lowercase{%
   \endgroup
       \let~\autoob
   }\begingroup
       \lccode`\~`]
       \lowercase{%
   \endgroup
       \let~\autocb
   }}

\delimiterfactor 1001

\makeatletter
% for amsmath "compatibility" (not sophisticated)
% \usepackage{amsmath}
\AtBeginDocument {%
          \def\resetMathstrut@{%
           \setbox\z@\hbox{\the\textfont\symoperators\char40}%
           \ht\Mathstrutbox@\ht\z@ \dp\Mathstrutbox@\dp\z@}%
}%
\makeatother
\author{Brendan Matthews}
\title{Linear Programming}
\begin{document}
\maketitle{}
\newpage{}
\section*{Well Known Problems}
\subsection*{Assignment Problem}
Minimise $\sum \sum c_{ij} x_{ij}$ so that $\sum_i x_{ij} = 1$ and $\sum_j x_{ij} = 1$.
\subsection*{Binary Knapsack}
Maximise $\sum c_j x_j$ so that $\sum a_j x_j \leq b$.
\subsection*{Set Covering Problem}
Let $M$ be a set of indices for regions and $N$ be a set of indices for stations.
Let $S_j \subset N$ be the indices of regions coverable
for cost $c_j$ by a station indexed by $j$.
Find $$\min_{J \subset N} \{\sum c_j :M = \bigcup_{j \in J} S_j \}.$$
Using an incidence matrix $A$ for our given covering options,
we can write this as $\min \sum c_j x_j$ so that $\sum a_{ij} x_j \geq 1$
\subsection*{TSP}
If $x_{ij} = 1$ then we go from town $i$ to town $j$. We want to minimise
$\sum_i \sum_{j \neq i} d_{ij} x_{ij}$ so that $\sum_{j \neq i} x_{ij} = 1$ (leave each town)
and $\sum_{i \neq j} x_{ij} = 1$ (each town is visited inc. starting point).
\begin{center}
\includegraphics[scale=0.6, angle=-0]{fig/IMG.png}
\end{center}
We don't want subtours so $\sum_{i \in S} \sum_{j \not \in S} x_{ij} \geq 1$ for each nonempty $S \subsetneq N$. \\
N.B. The TSP need not obey the triangle inequality, rocky terrain etc. If it does there are good polynomial time approximations.
\subsection*{Modelling disjunctions}
\begin{center}
\includegraphics[scale=0.6, angle=-90]{fig/IMG_7354.jpeg}
%\includegraphics[scale=0.6, angle=-90]{fig/IMG_7241.jpeg}
\end{center}
\begin{center}
\includegraphics[scale=0.3, angle=-0]{fig/projmax.png}
%\includegraphics[scale=0.6, angle=-90]{fig/IMG_7241.jpeg}
\end{center}
If the projection is nonempty so too are the sets in question, so their maximums exist (do they?).
Let $M$ be the second maximum and
let $(x_0,y_0) \in Q$ correspond to the argmax of the first set. Then
\begin{align*}
M &\geq \min_s u^s(b-Gy_0)+hy_0 \text{ since $y_0 \in P(Q)$} \\
&\geq u^s(Ax_0)+hy_0 \text{ since $(x_0,y_0) \in Q$}\\
&= (u^sA)x_0+hy_0 \text{ since matrix multiplication is associative} \\
&\geq cx_0 +hy_0 \text{ since $u^s \in U$.}
\end{align*}
%\begin{defn}
%The dual of a linear program
%\begin{gather*}
%\maxx {\bc}^{\top} \bx \\
%\st A\bx \leq \bb \text{ and } \bx \geq 0
%\end{gather*}
%\end{defn}
%is
%\begin{gather*}
%\minn {\bb}^{\top} \bu \\
%\st A^{\top}\bu \geq \bc \text{ and } \bu \geq 0.
%\end{gather*}
%\end{defn}
Fix $\by \in P(Q)$. The dual of the linear program
\begin{gather*}
\maxx \bc \bx \\
\st A\bx \leq \bb-G \by \text{ and } \bx \geq 0
\end{gather*}
%\end{defn}
is
\begin{gather*}
\minn \bu(\bb-G \by) \\
\st \bu A \geq \bc \text{ and } \bu \geq 0.
\end{gather*}
The objective value of the dual is found at an extreme point
and is equal to $u^{s(y)}(b-Gy)$ for some $s(y) \in \{1,\hdots,S\}$. In particular,
$cx_0+hy_0 = \max_y\{u^{s(y)}(b-Gy) + hy\} \geq M$ and so $cx_0+hy_0 = M$.
\end{document}

\begin{center}
\includegraphics[scale=0.6, angle=-90]{fig/IMG_7240.jpeg}
\includegraphics[scale=0.6, angle=-90]{fig/IMG_7241.jpeg}
\end{center}
