\documentclass{article}
\usepackage[utf8]{inputenc}
\usepackage{amsmath}
\usepackage{hyperref}
\usepackage{amsfonts}
\usepackage{listings}
\usepackage{color}
\definecolor{dkgreen}{rgb}{0,0.6,0}
\definecolor{gray}{rgb}{0.5,0.5,0.5}
\definecolor{mauve}{rgb}{0.58,0,0.82}
\lstset{frame=tb,
  language=Python,
  aboveskip=3mm,
  escapechar=\%,
  belowskip=3mm,
  showstringspaces=false,
  columns=flexible,
  basicstyle={\small\ttfamily},
  numbers=none,
  numberstyle=\tiny\color{black},
  keywordstyle=\color{black},
  commentstyle=\color{dkgreen},
  stringstyle=\color{black},
  breaklines=true,
  breakatwhitespace=true,
  tabsize=3
}
\usepackage{ mathrsfs }
\usepackage{amsthm}
\usepackage{amssymb}
\usepackage{graphics}
\usepackage{graphicx}
\usepackage{mathtools}
\usepackage{booktabs}
\DeclareMathOperator\dom{dom}
\DeclareMathOperator\vphi{\varphi}
\DeclareMathOperator\eps{\epsilon}
\DeclareMathOperator\del{\delta}
\DeclareMathOperator\Del{\Delta}
\DeclareMathOperator\lm{\lambda}
\DeclareMathOperator\deq{\vcentcolon=}
\DeclareMathOperator\R{\mathbb{R}}
\DeclareMathOperator\pr{\mathbb{P}}
\DeclareMathOperator\Z{\mathbb{Z}}
\DeclareMathOperator\N{\mathbb{N}}
\DeclareMathOperator\F{\mathbb{F}}
\DeclareMathOperator\A{\mathbb{A}}
\DeclareMathOperator\HH{\mathbb{H}}
\DeclareMathOperator\minn{\text{Minimise} \quad }
\DeclareMathOperator\maxx{\text{Maximise} \quad }
\DeclareMathOperator\st{\text{Subject to} \quad }
\DeclareMathOperator\nc{\text{no constraints}}
\DeclarePairedDelimiter\ceil{\lceil}{\rceil}
\DeclarePairedDelimiter\floor{\lfloor}{\rfloor}
\DeclareMathOperator\bx{\bold{x}}
\DeclareMathOperator\bs{\bold{s}}
\DeclareMathOperator\id{\text{Id}}
\DeclareMathOperator\bb{\bold{b}}
\DeclareMathOperator\bA{\bold{A}}
\DeclareMathOperator\bp{\bold{p}}
\DeclareMathOperator\bc{\bold{c}}
\DeclareMathOperator\C{\mathbb{C}}
\DeclareMathOperator\ran{ran}
\DeclareMathOperator\img{Im}
\DeclareMathOperator\op{\oplus}
\DeclareMathOperator\ot{\otimes}
\DeclareMathOperator\diam{diam}
\DeclareMathOperator\ite{int}
\DeclareMathOperator*{\argmax}{arg\,max}
\DeclareMathOperator*{\argmin}{arg\,min}
\DeclareMathOperator\cd{card}
\DeclareMathOperator\la{\langle}
\DeclareMathOperator\ra{\rangle}
\DeclareMathOperator\erf{erf}
\DeclareMathOperator\erfc{erfc}
\DeclareMathOperator{\sgn}{sgn}


\newcommand{\quo}{/ _\sim}
\newcommand{\pa}{\partial}
\newcommand{\lb}{\left}
\newcommand{\rb}{\right}
\newcommand{\aps}{\alpha_S}
\newcommand{\apl}{\alpha_L}
\newcommand{\tdd}{\frac{d^2T}{dx^2}}
\newcommand{\dx}{\frac{d}{dx}}
\newcommand{\seq}{(x_n)_{n \geq 1}}
\newcommand{\sseq}{(x_{n_k})_{k \geq 1}}
\newcommand{\fseq}{(f_n)_{n \geq 1}}
\newcommand{\elll}{\ell^{\infty}(\N)}
\newcommand{\norm}{{\|.\|}}
\newcommand{\inner}{\langle .,. \rangle}
\newcommand*\dd{\, \mathop{}\!\mathrm{d}}
\newcommand*\DD[1]{\, \mathop{}\!\mathrm{d^#1}}
\newcommand{\Ga}[1]{\frac{1}{\sqrt{2\pi\sigma^2}} e^{-\frac{\left(#1\right)^2}{2\sigma^2}}}
\usepackage{color}
%\title{}
\newcommand*\autoop{\left(}
\newcommand*\autocp{\right)}
\newcommand*\autoob{\left[}
\newcommand*\autocb{\right]}
\DeclareRobustCommand*\{{\ifmmode \left\lbrace \else \textbraceleft \fi }
\DeclareRobustCommand*\}{\ifmmode \right\rbrace \else \textbraceright \fi }
\AtBeginDocument {%
   \mathcode`( 32768
   \mathcode`) 32768
   \mathcode`[ 32768
   \mathcode`] 32768
   \begingroup
       \lccode`\~`(
       \lowercase{%
   \endgroup
       \let~\autoop
   }\begingroup
       \lccode`\~`)
       \lowercase{%
   \endgroup
       \let~\autocp
   }\begingroup
       \lccode`\~`[
       \lowercase{%
   \endgroup
       \let~\autoob
   }\begingroup
       \lccode`\~`]
       \lowercase{%
   \endgroup
       \let~\autocb
   }}

\delimiterfactor 1001

\makeatletter
% for amsmath "compatibility" (not sophisticated)
% \usepackage{amsmath}
\AtBeginDocument {%
          \def\resetMathstrut@{%
           \setbox\z@\hbox{\the\textfont\symoperators\char40}%
           \ht\Mathstrutbox@\ht\z@ \dp\Mathstrutbox@\dp\z@}%
}%
\makeatother
\author{Brendan Matthews}
\title{Monty Carlo Book}
\begin{document}
\section{A FEW BASIC TERMS IN PROBABILITY AND STATISTICS}
Covariance of two functions is the difference of expected value of product and product of expected values.
Correlation is that over the geometric mean of their variances. The nth moment of a dist $\psi$ is $$
E_{\phi}[x^n] = \int x^n \psi(x)dx.
$$
Jensens Inequality for convex functions is $$
E_{\psi}[g(X)] \geq g(E_\psi[X])
$$
Holders inequality for any two r.v. $X,Y$, $p>1$ is $$
E[XY] \leq \| X \|_p \|Y\|_{\frac{p}{p-1}} = \sqrt[p]{E(|X|^p)} \times \hdots
$$
Minkowski's Inequality is the triangle for $X,Y$ and $p \geq 1$.
\section{Monte Carlo Simulations}
Suppose for $x \in \R^n$ we need to calc $$
v = E_{\psi(x)}(f(x)) = \int f(x) \psi(x)dx^n.
$$
The Monte Carlo estimator is then $$
\hat{v}_N = \frac{1}{N} \sum^N f(x_i).
$$
\subsection{Monte Carlo Supremacy}
\begin{center}
\includegraphics[scale=0.5]{myl4pic1.png}
\includegraphics[scale=0.5]{myl4pic2.png}
\includegraphics[scale=0.5]{myl4pic3.png}
\end{center}
\section{Common Distributions}
\subsection{Uniform}
We say that $X \sim \mathcal{U}[a,b]$ or $X \sim \mathcal{U}(a,b)$ with pdf $$
\psi(x) = \frac{1}{b-a} \times \{x \in D(X)\},
$$
where curly brackets denote binary logical outcomes.
\subsection{Normal}
We say $X \sim \mathcal{N}(\mu,\sigma)$. The pdf is $$
\psi(x; \mu,\sigma) = \Ga{x-\mu}.
$$
If $\mu = 0$ as is the case for the std normal dist, odd moments are zero by symmetry about origin.
For the std normal we have the nice result for even moments using IBP on $x^m e^{-g(x^2)}$ as $x^{m-1}(xe^{-g(x^2)})$ we have
$$
E(X^{2k}) = \frac{2^k}{\sqrt{\pi}} \Gamma(k+0.5) = \frac{(2k-1)!}{2^{k-1}(k-1)!}.
$$
For the cdf, there holds $\erf(x) = 2N(x\sqrt{2})-1$.
\subsection{Binomial}
The probability of $k$ successes in $n$ trials is
$$
{n \choose k}p^k(1-p)^{n-k}
$$
and pdf for Binomial variate $X$ is $$
\psi(x) = \sum_{k=0}^n {n \choose k} p^k(1-p)^{n-k}\del(x-k)
$$
where $\del$ is Dirac. The cdf is
$$
\int_{-\infty}^x \sum_{k=0}^n {n \choose k} p^k(1-p)^{n-k}\del(x-k)dx =  \sum_{k=0}^n {n \choose k} p^k(1-p)^{n-k} \int_{-\infty}^x \del(x-k)dx =  \sum_{k=0}^n {n \choose k} p^k(1-p)^{n-k}H(x-k)
$$
where $H$ is Heaviside.
\subsection{Geometric}
Number of Bernoulli trials until first failure. The pdf is $$
\psi(y) = \sum^{\infty}_{\color{red}{k=1}} p^{k-1}(1-p) \del(y-k)
$$
and the cdf is Heaviside.
\subsection{Poisson}
If $\lm$ is the avg rate at which events occur, i.e. $dp = \lm dt$ then the number of events occuring in period $T$ has pdf
$$
\psi(y) = \sum_{k=0}^\infty \frac{(\lm T)^ke^{-\lm T}}{k!}\del(y-k) = \sum_{k=0}^\infty \frac{\lm^ke^{-\lm}}{k!}H(y-k)
$$
and the cdf is Heaviside.
\subsection{Beta}
Note that
$$
B_x(\alpha,\beta) = \int_0^x t^{\alpha-1} (1-t)^{\beta-1} dt
$$
is the incomplete beta function and the complete has $x=1$.
With equal shape parameters, the beta
distribution is symmetric, skewness is 0 and mode, mean, median are one half.
The pdf is
$$
\frac{x^{\alpha-1} (1-x)^{\beta-1}}{B(\alpha,\beta)} = \frac{\Gamma(\alpha+\beta)}{\Gamma(\alpha)\Gamma(\beta)}x^{\alpha-1} (1-x)^{\beta-1}
$$
and the cdf is $$
\frac{B_x(\alpha,\beta)}{B(\alpha,\beta)}.
$$
We have
\begin{center}
\includegraphics[scale=0.5]{myl4pic4}
\end{center}
\subsection{Gamma}
The incomplete Gamma function is $$
\Gamma(z,x) = \int_x^\infty t^{z-1}e^{-t}dt \text{ and the gamma function is $\Gamma(z) = \Gamma(z,0).$}
$$
In the following, $\color{red}{x},\alpha,\beta \geq 0$.
The pdf is $$
\psi(x) = \frac{x^{\alpha-1}}{\Gamma(\alpha)\beta^\alpha}e^{-\frac{x}{\beta}}
$$
and integration by substitution gives  the cdf
$$
\frac{1}{\Gamma(\alpha)}\int_{-\infty}^{x \slash \beta} \beta \psi(\beta t)dt =
\frac{1}{\Gamma(\alpha)}\int_{0}^{x \slash \beta} \beta \psi(\beta t)dt =
$$
$$
\frac{1}{\Gamma(\alpha)}\int_{0}^{\infty} \beta \psi(\beta t)dt -
\frac{1}{\Gamma(\alpha)}\int_{x \slash \beta}^\infty \beta \psi(\beta t)dt =
1-\frac{\Gamma(\alpha, x \slash \beta)}{\Gamma(\alpha)}.
$$
Notice that the pdf is defined in such a way to cancel out $\Gamma(\alpha)$ at the lower limit of the cdf and $\Gamma(\alpha, \infty) = 0$.
For $\alpha = n$ and $\beta = 1 \slash \lm$, the gamma dist is the amount of time for $n$ events to occur in a Poisson process.
\subsection{Chi Square}
This is the distribution of the sum of squares of indep std normal variates with $\nu$ terms (degrees of freedom). Its important
in hypothesis testing for variance. It is a special case of the gamma distribution with $\alpha = \nu \slash 2$, $\beta =2$.
\subsection{Students $t$}
This distribution arises when we seek to test the difference of
means when underlying variance is not known.
When a standard normal variate
$$
Z = (\bar{X}_n - \mu) \frac{\sqrt{n}}{\sigma}
$$
has stdev replaced by bias corrected $s$,
the students $t$ variate $$
T = (\bar{X}_n - \mu) \frac{\sqrt{n}}{s}
$$
emerges. It happens that for $V = (n-1)\frac{s^2}{\sigma^2}$, a Chi-squared variate, $$
T = \frac{Z}{\sqrt{V \slash \nu}}.
$$
Since the expectation of $V$ is $\nu$, $T$ approaches $Z$ in distribution as the sample size increases.
\subsection{Standard Cauchy Distribution}
This is a student's $t$ with one degree of freedom. The pdf is $$
\frac{1}{\pi(1+x^2)}.
$$
Its symmetric about zero so odd moments are zero. Even moments don't exist. It is one of the few stable distributions with a probability density function that can be expressed analytically, the others being the normal distribution and the Lévy distribution.
\subsection{Lognormal}
Take the exponential of the product of a std normal variate with a parameter $\sigma$.
The pdf is $$\frac{1}{x}\Ga{\ln(x)}$$ and the cdf is $$
N(\frac{\ln(x)}{\sigma})
$$
where $N$ is the std normal cdf. In fact there can be the parameter $\mu$ added in the exponential but we set it to zero here. If it were present, the gaussian would feature $(\ln(x)-\mu)^2$. A log-normal process is the statistical realization of the multiplicative product of many independent random variables, each of which is positive. This is justified by considering the central limit theorem in
the log domain. \\ \newline
The lognormal distribution is probably the most important function in computational
finance since it is the solution to the stochastic differential equation describing geometric
Brownian motion. which in turn is almost always the first easy choice when the evolution
of non-negative quantities is to be modelled.
\subsection{Generalised Beta 2}
The generalised beta distribution of the second kind
is given by a four-parameter density function. One of the very useful features of the GB2 distribution is that it allows for a large variety
of shapes that are nearly lognormal, which is desirable when density functions are used
for the extrapolation of volatility smiles of traded options.
\subsection{Pareto}
The Pareto distribution is skewed and heavy-tailed. It is used to model rare events. The pdf is $$
\frac{ab^a}{x^{a+1}}
$$
where $x \geq b, a > 0$. The moments are $$
E(X^k) = \frac{ab^k}{a-k}
$$
for $a > k$.
\end{document}
